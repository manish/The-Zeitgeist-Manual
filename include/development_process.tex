\section{How we handle bugs}
\section{How we handle blueprints}
\section{How we handle merge requests}

All important changes needs to be reviewed before it lands in our source 
code repository. Anyone who wants to make a change, which includes the 
developers too, have to propose a merge request. The changed in the merge 
request is then reviewed by other developers before it lands in the source 
code repository. As a general policy, atleast one developer needs to approve 
the merge request before the request is considered accepted by the team.

There are various advantages of reviews using merge requests which includes, 
but is not limited to the following reasons

\begin{itemize}
\item It helps find the probable issues which might be encountered if the 
changes proposed contains bugs.
\item It helps the person proposing the merge request to learn something new 
when one of the developers proposes a better solution than the one provided.
\item It helps the team learn any new practices or feature which would 
otherwise go un-noticed if the changes were directly pushed in the main 
repository without a review
\item It helps the team to keep a track on the changes occouring in the source 
code repository.
\end{itemize}

You can relate this process with peer-reviews of scientific journals. Instead of 
new theories, changes are being proposed. The scientific community's role is being 
played mostly by the developers (though anyon can leave a review). If the changes 
is found to meet the requirements and expecations, they are accepted in the 
codebase just like theory is accepted in the scientific knowledge repository.

\section{How we test our work}
\section{How we manage releases}
A release is discussed between the various module maintainers. 
After the discussion the goals for the upcoming release is set 
and a release date is decided. This release date date is a vague 
estimation and can change depending on various factors like 
feature set, number of bugs, manpower availability etc. 
Depending on the feature sets and bugs as well as the personal 
availability a release manager is assigned to take the lead 
on driving the release.

There are a few things to be kept in mind:
\begin{itemize}
\item Most of the releases are taken care by a single release manager 
with rest of the team members assisting him. 
\item The selection of release manager has no fixed criteria. The 
person who does the most amount of work in a particular cycle takes 
up the role.
\item Release manager is not a status symbol, it is a responsibility.
\item We usually switch roles from developer to tester to 
release manager to keep everyone refreshed.
\item Switching roles and having different release manager for releases 
keeps us from getting burnt out.
\item Every release has a name. The release manager is the person who 
picks up the name. He can consult other team members for suggestions.
\end{itemize}

\section{Description of our development process}
\section{Governance model}

The zeitgeist project is a mix of meritocratic and democratic practises. 
When an important topic is discussed, all the members involved are 
consulted and the best decision is carried forward. The experience and 
knowledge of a member related to the topic carries weight and the decision 
is based on it.

There are components which have a single maintainer. Those maintainers usually 
take decisions which are in best interest of the project or to that component. 
Maintainers sometimes consult other members in cases when the impact of the 
decision might be noticable. 

Components which are maintained by a group of developers are mostly important 
ones and the decision is taken largely by concensus. The person who took more 
active part in the ongoing version gets a stronger voice and he/she knows the 
changes better than others who are passively participating. 