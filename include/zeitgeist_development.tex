\section{Introduction}

As mentioned before Zeitgeist is an event log and not a search engine or 
file tracker. It just logs the various events occouring on the system 
like file opened, file modified, call placed, IM received etc.

\section{Engine}

At the heart zeitgeist consists of an engine which runs as a daemon. It's 
work is to simply receive events, store it in the database, provide the 
events when requested for. The daemon exposes a per-user D-Bus which runs 
as a SessionBus.


\noindent
\linebreak 
The daemon stores all the events and related metadata in an sqlite database 
which is stored in 

\texttt{\$XDG\_DATA\_HOME/zeitgeist/activity.sqlite}

\noindent
where \texttt{\$XDG\_DATA\_HOME} equals to \texttt{\$HOME/.local/share} in 
most of the cases.


\noindent
\linebreak 
If there are more than one user on the system, then every user will have 
their own Zeitgeist SessionBus and thus their own seperate sqlite database. 
This behaviour of SessionBus is useful for keeping activities of different 
users stored under their own directory.

\section{The Basic Unit: Event}

If the engine is the heart, then the role of blood is being taken up by Events. 
Events are packet of information which contain the data about any kind of activity 
occouring on the system.


\noindent
An Event contains:
\begin{itemize}
\item \textbf{id} - Every event has a unique identifier which can be used to identify 
an event from others. It is of type int
\item \textbf{timestamp} - This is the time when the event occoured. The value stored is 
the number of milliseconds since UNIX epoch. UNIX epoch is the moment which is 
00:00:00 UTC on Thursday, January 1st, 1970.
\item \textbf{interpreation} - This defines "what happened" using a formal URI. The 
predefined set of URIs are available and will be covered later.
\item \textbf{manifestation} - This defines “how did this happen” using a formal URI. 
The predefined set of URIs are available and will be covered later.
\item \textbf{actor} - The URI defining the entity responsible for this event. 
In most of the case it is an application. In case of application the 
application:// URI is used. e.g. application://firefox.desktop
\item \textbf{subjects} - This just represents the subject of the event. There can be more 
than one subjects but in majority of the cases, only one is needed
\item \textbf{payload} - An array of bytes which can be used for freeform storage of any 
data associated with the event. The encoding or format of the data is not handled 
by the engine and is left upon the clients.
\end{itemize}


\noindent
A Subject contains:
\begin{itemize}
\item \textbf{uri} - The Universal Resource Identifier associated with the resource involved 
in this event. This URI stored should be such that the event it is representing can 
be reproduced using this. e.g. In case a file is opened, closed, edited then the URI 
will be file URL which can be like \texttt{file:///etc/fstab} or if the webpage was 
accessed, then the URL of the website becomes the URL like \texttt{http://google.com} 
or if the application is launcher, then \texttt{application://firefox.desktop} 
can be the URI if Firefox was launched.
\item \textbf{interpreatio}n - the abstract notion of what the subject is, eg. \"this is a document\" or \"this is an image\". The interpretation is formally represented by a URI.
\item \textbf{manifestation} - the abstract notion of how the subject is stored or available, eg. \"this is file\" or \"this is a webpage\". The manifestation is formally represented by a URI.
\item \textbf{origin} - The URI from where the user accessed the subject from. In case the user 
opened the file \texttt{file:///etc/fstab}, then the origin is \texttt{file:///etc/} and 
in case of \texttt{http://google.com/accounts/list} origin is \texttt{http://google.com/} 
and in case of application launches origin is empty.
\item \textbf{mimetype} - The mimetype of the subject. Examples are \texttt{text/plain}, 
\texttt{text/html}, \texttt{image/png} and \texttt{application/octet-stream}
\item \textbf{text} - A short textual representation of the subject. In case of PDF, the title of 
the PDF can be the text. In case of music track - Title, Genre and artist is the text. In 
case of webpage accessed, the title of webpage can be the text.
\item \textbf{storage} -  represents the id storage device the subject is. For files this would be the UUID of the volume, for subjects requiring a network interface use the 
string “net”. If the subject has been deleted use the string “deleted”. The storage id 
of the subject is used internally in the Zeitgeist engine to keep track of subject 
availability. This way clients can request hits only on subjects that are 
currently available.
\end{itemize}

\section{The D-Bus interface}

Zeitgeist exposes a D-Bus Session bus with the busname as \texttt{org.gnome.zeitgeist.Engine}, 
Object path as \texttt{/org/gnome/zeitgeist/log/activity} and interface 
\texttt{org.gnome.zeitgeist.Log}


The following operations are possible over the interface

\begin{itemize}
\item Inserting events
\item Deleting Events
\item Fetching events by passing the event ids
\item Searching events by passing the template which would be matched against events
\item Exiting the daemon. It is equivalent of calling quit
\end{itemize}

Most of the time you would never need to deal directly with D-Bus since zeitgeist has 
libraries for most of the famous languages which abstract the complexity of D-Bus. We 
have language bindings for C, Vala, Python, C++/Qt and C\#. Each binding abstracts the 
complexity and presents a developer friendly frontend and the programming patterns 
for individual languages.

\section{Communicating over D-Bus}

Even though you would not need to communicate directly over D-Bus, it would be a good 
way to learn by communicating directly. We would be using python as the language, since 
you don't need to write Makefiles and compile it before running.

Please make sure you have Python and zeitgeist installed. You can install it from your 
distribution's package management tool. You would need to access the python shell for 
which you can use the built-in shell but you can use bpython, ipython or any other 
of your choice.

Open, terminal and type \texttt{python} which will lead you to a prompt looking like 
\texttt{>>>}

\subsection{Fetching events by id}

Now you need to import the relevant packages and types. We would first try to get an 
instance of the D-Bus interface, then we would invoke the methods.

\begin{center}
\begin{verbatim}
>>> import dbus

# Create an instance of the SessionBus
>>> bus = dbus.SessionBus()

# Fetch the Object Path of Zeitgeist
>>> z = bus.get_object("org.gnome.zeitgeist.Engine", \
          "/org/gnome/zeitgeist/log/activity")

# Get the interface containing all the methods and signals
>>> zg = dbus.Interface(z, "org.gnome.zeitgeist.Log")
\end{verbatim}
\end{center}

You can now consider \texttt{zg} as an object over which you can invoke various 
methods. You can either search for events which acts like a filter based on certain 
criteria or you can fetch events based on event ids. We will try the latter first 
using arbiturary event ids to get a grasp of the methods. then we will learn how to 
fetch the events or event ids

\begin{center}
\begin{verbatim}
# Fetch event with id 1,2 by calling method GetEvents
# ev12 is an arry which contains event instances
>>> ev12 = zg.GetEvents([1,2])

# Check if such an event is available
>>> ev12[0] is not None
True

# We now try to create Event objects
>>> from zeitgeist.datamodel import Event
>>> e1 = Event(ev12[0])

# You might get different values
>>> e1.get_actor().__str__()
'application://compiz.desktop'

>>> e1.get_interpretation().__str__()
'http://www.zeitgeist-project.com/ontologies/2010/01/27/zg#AccessEvent'

>>> e1.get_manifestation().__str__()
'http://www.zeitgeist-project.com/ontologies/2010/01/27/zg#UserActivity'

# you get the milliseconds since EPOCH
>>> e1.get_timestamp().__str__()
'1316814635513'

>>> secs = int(e1.get_timestamp().__str__())/1000

>> import datetime
>>> datetime.datetime.fromtimestamp(float(secs))
datetime.datetime(2011, 9, 24, 3, 20, 35)

# Now we want to fetch subjects
>>> subs=e1.get_subjects()
>> len(subs)
1

>>> sub = subs[0]

>>> sub.get_current_uri().__str__()
'application://gnome-activity-journal.desktop'

>>> sub.get_interpretation().__str__()
'http://www.semanticdesktop.org/ontologies/2007/03/22/nfo#Software'

>>> sub.get_manifestation().__str__()
'http://www.semanticdesktop.org/ontologies/2007/03/22/nfo#SoftwareItem'

>>> sub.get_mimetype().__str__()
'application/x-desktop'

>>> sub.get_origin().__str__()
''

>>> sub.get_storage().__str__()
''

>>> sub.get_text().__str__()
'Activity Journal'

>>> sub.get_uri().__str__()
'application://gnome-activity-journal.desktop'


\end{verbatim}
\end{center}

\subsection{Searching for events by template}

In this section we will learn to search for events based on a certain set of filters.
The template used for filtering is an instance of an Event itself.

Even though it looks tougher with all the type conversions, all these appears in 
our snippet because we are using raw D-Bus. In case of libraries all these 
boilerplate code is hidden by a friendly API

\begin{center}
\begin{verbatim}

# Import the necessary types
>>> from zeitgeist.datamodel import TimeRange, StorageState, ResultType

# Create an instance of an event template
>>> tmp = Event.new_for_values(actor="application://gedit.desktop")
# We cast this object into a list, so that it can be sent over D-Bus
>>> tmp_list = list(tmp)
>>> all_tmps = [tmp_list]

# Specify the time range. In this case we choose to get all events till now
>>> tr = TimeRange.until_now()
>>> tr_list = list(tr)

# We are asking for events for which subjects are available 
# and sorted by Most Recent Events and total of 6 events
>> ids = zg.FindEventIds(tr_list, all_tmps, StorageState.Available, \
   6, ResultType.MostRecentEvents)

# Now we can get the event ids. e.g.
>>> ids[0].__int__()
99738

# If you want to cross check that the event you got indeed has actor 
# as you specified, you can try to fetch the whole event
>>> ev=zg.GetEvents([int(ids[0])]
>>> Event(ev[0]).get_actor().__str__()
'application://gedit.desktop'



\end{verbatim}
\end{center}

In the previous section, we saw that we are searching for events which have 
the value of actor set as \texttt{application://gedit.desktop} and the called 
method is \texttt{GetEventIds}. The method \texttt{GetEventIds} is useful in 
cases only if you want to know only the ids and not interested in all the events. 
This can be particularly useful when you don't want to store all the events 
in memory at once, but would like to fetch the event only when needed.


In most practical scenarios, you would send an event template and fetch the 
events itself instead of just their ids. To accomodate for this workflow, you 
don't need to make drastic changes in the code snippet in the previous example.


Instead of \texttt{FindEventIds}, you would use \texttt{FindEvents} to fetch 
the events. Both the methods use the same arguments.

\begin{center}
\begin{verbatim}

>>> events = zg.FindEvents(tr_list, all_tmps, StorageState.Available, \
   6, ResultType.MostRecentEvents)

>>> ev1 = Event(events[0])
>>> ev1.get_actor().__str__()
'application://gedit.desktop'

\end{verbatim}
\end{center}

This methods can be viewed as calling \texttt{FindEventIds} and then calling 
\texttt{GetEvents} using the returned list of event ids as it's arguments.

\subsection{Inserting Events}

Till now we have learnt how to fetch events from the data store by using templates 
which acts as a filter. In this section we will learn how to insert events.

\begin{center}
\begin{verbatim}

>>> from zeitgeist.datamodel import Event, Subject
>>> ev = Event.new_for_values(actor="application://firefox.desktop")
>>> sub = Subject.new_for_values(uri="application://firefox.desktop")
>>> ev.subjects.append(sub)



\end{verbatim}
\end{center}

\section{Using the Python API}

In the previous section we learnt how to use Zeitgeist API by directly calling 
the methods over D-Bus. The method invocation in the examples were blocking in nature 
which is not useful in real situations. 

Now we will learn how to use the official Zeitgeist Python API to use Zeitgeist 
without having to worry about the D-Bus methods and the underlying technology. These 
methods are asynchronous in nature. Every method call needs to provide a callback 
method to which the control is passed after the operation completes. This is 
especially useful in cases when the user is performing incremental search where 
the call to zeitgeist is made after every keystroke.

\subsection{Searching for events which matches the template}

In the following examples we won't be importing dbus and try to get the bus instance 
but instead use the python wrapper which does all the heavy lifting for us.

You can save this in a file and run it on command line using python command passing 
the filename as the argument

\texttt{\$ python file.py}

\begin{center}
\begin{verbatim}

from gi.repository import GObject
from zeitgeist.client import ZeitgeistClient
from zeitgeist.datamodel import Event, Subject, Interpretation, ResultType
zg = ZeitgeistClient()

event_template = Event()
subj = Subject()
subj.interpretation = Interpretation.VIDEO
event_template.subjects = [subj]

mainloop = GObject.MainLoop()

def callback(events):
	for ev in events:
		print ev.subjects[0].get_uri()
	mainloop.quit()

zg.find_events_for_templates([event_template], callback, \
    num_events=100, result_type=ResultType.MostPopularSubjects)
mainloop.run()

\end{verbatim}
\end{center}

Now let's see what we are doing

\begin{enumerate}
\item We import GObject so that we can use it as the mainloop
\item Then we import ZeitgeistClient which is a wrapper around the D-Bus interface 
which abstracts the implementation level detail and provides us with an asynronous 
API for real world usage
\item We import the data stuctures which will contain the information we would 
send to the zeitgeist engine
\item We create an instance of the Client named "zg"
\item We create an instance of Event and Subject. We set the interpretation of the 
subject to the VIDEOS and add the subject to the event subjects list
\item We create an instance of GObject Mainloop named "mainloop"
\item We create a callback named "callback" which accepts an argument named events 
which is the information returned from the zeitgeist engine
\item We invoke the method \texttt{find\_events\_for\_templates} with the event\_template, 
callback, maximum number of events to be fetched and ResultType as arguments
\item The value MostPopularSubjects tells the engine to return one event for each 
subject and sort it based on popularity of the Subject.
\end{enumerate}

\subsection{Fetching events from event ids}

In this example we will fetch the event instances when we already know the event ids

\begin{center}
\begin{verbatim}

from gi.repository import GObject
from zeitgeist.client import ZeitgeistClient
from zeitgeist.datamodel import Event, Subject, Interpretation
zg = ZeitgeistClient()

mainloop = GObject.MainLoop()

def callback(events):
	for ev in events:
		print ev
	mainloop.quit()

zg.get_events([1,2], callback)
mainloop.run()

\end{verbatim}
\end{center}

In this case we don't need to create Event templates as we already know the event ids 
which are 1 and 2. We have a callback method which takes a single argument which is 
an array of events returned for the event ids provided.

\subsection{Searching for event ids which matches the template}

There are cases when you want to know the events which match a specific template, 
but don't want to fetch all such events because you would use lazy loading for showing 
the event details or it does not make sense to show everything since not everything 
can show at once on the screen.

For such situations, it is best to get all the event ids for events which match 
the template and call \texttt{get\_events} only when such event details needs to be shown.

\begin{center}
\begin{verbatim}

from gi.repository import GObject
from zeitgeist.client import ZeitgeistClient
from zeitgeist.datamodel import Event, Subject, Interpretation
zg = ZeitgeistClient()

event_template = Event()
subj = Subject()
subj.interpretation = Interpretation.VIDEO
event_template.subjects = [subj]

mainloop = GObject.MainLoop()

def callback(ids):
	for id in ids:
		print id
	mainloop.quit()

zg.find_event_ids_for_templates([event_template], callback, num_events=100, result_type=0)
mainloop.run()

\end{verbatim}
\end{center}

\subsection{Inserting Events}

In this section we will learn how to insert events in the zeitgeist log. We would need 
to create the Event instance and fill it up with necessary information and then send it 
asynchronously over d-bus.

\begin{center}
\begin{verbatim}

from gi.repository import GObject
from zeitgeist.client import ZeitgeistClient
from zeitgeist.datamodel import Event, Subject, Interpretation, Manifestation
zg = ZeitgeistClient()

event = Event()
event.interpretation = Interpretation.ACCESS_EVENT
event.manifestation = Manifestation.USER_ACTIVITY
event.actor = "application://vlc.desktop"

subj = Subject()
subj.interpretation = Interpretation.VIDEO
subj.manifestation = Manifestation.FILE_DATA_OBJECT
subj.uri = "/home/manish/Videos/Darkness.ogg"
subj.mimetype = "video/ogg"
subj.text = "Darkness exposed"
event.subjects = [subj]

mainloop = GObject.MainLoop()

def callback(ids):
	for id in ids:
		print id
	mainloop.quit()

zg.insert_event(event, callback)
mainloop.run()

\end{verbatim}
\end{center}

\noindent
In the above example we are inserting a specific event. Let's discuss

\begin{enumerate}
\item We import the necessary modules like ZeitgeistClient etc.
\item We create an instance of Event with Interpretation as ACCESS\_EVENT which 
means that the file was being accessed(read)
\item The Manifestation of the event was USER\_ACTIVITY as the event was caused due 
to user's activity. Other kind of manifestation can be SCHEDULED\_ACTIVITY in case 
when a new track starts playing because it was present in the playlist. There is 
also WORLD\_ACTIVITY which is used to represent activities which happen around the 
user which has not been created by the user like network disconnections which 
occour due to reasons out of control of the user
\item The actor is set to VLC as this application was used to open the file
\item Then we create a Subject which deals with the resource used in the event.
\item We set the subject interpretation to VIDEO as the file is of the type VIDEO
\item We set the subject manifestation as FILE\_DATA\_OBJECT as the subject is 
actually a file on the disk
\item Then we set the uri of the subject which is actually a file
\item We set the subject mimetype. In this case the video is of type video/ogg
\item We set the subject text which is a free form string describing the subject
\item We create the necessary mainlopp, callback and then call the event insert\_event 
with the event  we created and the callback.
\item Whent he mainloop is run, the event is inserted and the callback is invoked 
which in turn prints the event id of the inserted event
\end{enumerate}

The callback argument is list containing the event id of the 
inserted event. You can also insert multiple events at once using 
\texttt{insert\_events} which would accept the same callback 
but the callback argument which is a list would contain more than 
one item. The first argument of \texttt{insert\_events} is a list 
of events instead of a single event.